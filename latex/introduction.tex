\chapter{Introduction}
\section{Motivation}
Computer generated music, such as synthsized MIDI, are often considered robotic and unexpressive. But we have already witnessed the fluid and lively sound generated by state-of-the-art text-to-speech systems. This inspired us to develop a system that can read a music score and play it in an expressive, humanly way. Such system can be used for audiolizing score notation editing software, creating interactive media content, and generating royality-free music. 

Established pianists always has his/her own distinctive style. Such sytle distinguised himself/herself from all the other pianists. If the expressive performance system can learn the style of a performer, it might be able to provide musicological insight of performance styles. Furthermore, we can even make a mastero who is no longer with us play music he/she never played in his/her lifetime.

%Application: musicology study, typesetting tool, play score archive, play computer-generated music, accompaniment

\section{Previous Works}
%TODO: Previous Works
%TODO: Music Robot => No expressive
%TODO: Mitsu => No expressive
%TODO; 2008 Paper
%TODO: 2013 book
%TODO: recon



%Evaluation is difficult because:
%   * only RenCon
%   * Different level of automation/human intervention
%   * Different goals: reproduce human, expressive, creative
%
%Rule based:
%   Pros:
%      *  Large room for creativity, because there are a lot of parameters to tune
%      * polyphonic rule is possible
%   Cons
%      * Can only capture rough concept, not the nuences
%      * Expert interveiw can only get rules that the experts are aware of, not the unconscious ones
%      * When there is a rule, there is an exception
%      * Manual analysis is mendatory
%
%Linear Regression
%   Pros
%      * Simple
%   Cons
%      * Features are not linear




\section{Contribution}
The major contribution of this thesis is appling structural support vector machine on expressive performance problem. Also we tried to provide a public corpus for expressive performance. 
%TODO: scope
%TODO: new method for old problem
%\bibliographystyle{unsrt}
%\bibliography{thesisbib}
