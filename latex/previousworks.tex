\chapter{Previous Works}
\label{chap:prev}
\section{Various Goals and Evaluation}
The general goal of a computer expressive performance system is to generate expressive music, as opposed to the robotic and dull expression of rendered MIDI. But the definition of "expressive" is very vague and ambiguous, so each research will need to define a more precise and measurable goal. The following are the most popular goals a computer expressive performance system wants to achieve:
\begin{enumerate}
   \item Perform music notations in a non-robotic way, regardless of the style.
   \item Reproduce a human performance or a musician's style.
   \item Accompany a human performer.
   \item Validate a musicological theory of expressive performance.
   \item Directly render computer composed music works.
\end{enumerate}

%Some systems try to perform music notations in a non-robotic way in a general sense, without a certain style in mind. These systems has been employed in music typesetting softwares, like Sibelius \cite{sibelius}, to play the notation expressively. Most systems will implicitly achieve this goal.
%TODO:discuss the goals

%Systems that are designed to reproduce certain human performance or style are usually designed or trained using a particular performer's recording as reference. One commercial example is the Zenph re-performance CD\cite{zenph}, which reproduced the performance style of Rachimaninov, it can perform pieces that Rachimaninov never played in his lifetime in his style. 
%
%
%Accompaniment systems try to render expressive music that act as an accompaniment for a human performer. The challenge is that the system must be able to track the progress of a human performance and render the accompaniment in real-time. One commercial example is Cadenza\cite{cadenza}, using the technology created by Christopher Raphel\cite{chris}. It claims that it can help music student practice concertos with ease.
%
%Another goal is to validate musicological theories. Musicologist may propose theories on expressive music performance, some of them may want to build a generative model to validate their assumptions. These systems may focus more on the specific phenomenon that the theory tries to explain instead of generating music that is pleasant to human. 
%
%Finally, some systems combines computer composition with expressive performance. These systems have a great advantage because the intention of the composition can be shared with the performance module. Other systems that performs past compositions can only guess the composer's intention by analyzing the score notation. These systems usually has their own data structure to represent music, which can contain more information than traditional music notation, but the resulting performance system is not backward compatible with past compositions.

%Each system
\section{Researches Classified by Methods Used}
%\begin{table}
%   \centering
%   \begin{tabular}{|c|c|}
%      TBD & TBD\\
%      TBD & TBD\\
%   \end{tabular}
%   \caption{List of Reviewed Systems}
%   \label{tab:prevworks}
%\end{table}
Despite the difference between goals of different expressive performance systems, all expressive performance systems must have some strategy to learn and apply performance knowledge. There are generally two approach: rule-based or machine learning-based.

Using rules to generate expressive music is probably the earliest approach. Director Musices \cite{17} is one of the early examples.   Pop-E \cite{28} is also a rule-based system which can generate polyphonic music, using its synchronization algorithm to synchronize voices. Computational Music Emotion Rule System \cite{31} tried to develop rules that express human emotions.  Other systems like Hierarchical Parabola System \cite{17}\cite{18}\cite{19}\cite{20}, Composer Pulse System\cite{21,22}, Bach Fugue System\cite{23}, Trumpet Synthesis System \cite{24, 25} and Rubato \cite{26, 27} are also some systems that use rules to generate expressive performance. Most of the systems focus on expressive attributes like note onset, note duration and loudness, but Hermode Tuning System \cite{29} put special emphasis on intonation. Rule-based systems are generally more computationally efficient because the mathematical model is much simple than those learned by machine learning algorithms. And rules are generally more understandable to human than complex model parameters. But some of the nuance, such as some subconscious deviation or grouping of notes, may be hard to describe by rules, so there is a natural limit on how complex the rule-based system can be. Lack of creativity is also a problem for rule-based approach.

Another approach is to acquire performance knowledge by machine learning. Many machine learning methods have already been applied to this problem. For example, Music Interpretation System \cite{32,33,34} and CaRo \cite{35,36,37} both use linear regression to learn performance knowledge. But it is very unlikely that the expressive performance problem is a linear system, so Music Interpretation System try to solve it by using AND operations on linear regression results to handle non-linearity. But linear regression is still an oversimplification for such problem.

More complicated algorithms have also been applied: ANN Piano \cite{38} and Emotional flute \cite{39} uses artificial neural network. ESP Piano \cite{55} and Music Plus One \cite{52,53,54} uses Statistical Graphical Models such as Hidden Markov Model (HMM) and Bayesian Belief Network, but they did no use structural support vector machine to train the HMM.%, although the later one focus more on accompaniment task rather than rendering notation. 
KCCA Piano System \cite{57} uses kernel regression. Drumming System \cite{82} tried different mapping models that generates drum patterns.

Evolutionary computation such as genetic programming is used in Genetic Programming Jazz Sax \cite{88}. Other examples include the Sequential Covering Algorithm Genetic Algorithm\cite{59}, Generative Performance Genetic Algorithm \cite{89} and Multi-Agent System with Imitation \cite{60, 93}. Evolutionary computation takes long training time, and the results are unpredictable. But unpredictable also means there are more room for performance creativity, so these system can create unconventional but interesting performances.

\framebox{TODO: Discuss works that focus on timber only, e.g. Prof. Su's violin work}
Another approach is to use case-based reasoning. SaxEx\cite{40,41,42} use fuzzy rules based on emotions to generate Jazz saxophone performance. Kagurame \cite{43,44} focus on style (Baroque, Romantic, Classic etc.) instead of emotion. Ha-Hi-Hun \cite{45} has a more ambitions goal in mind: to accept natural language instructions like \enquote{Perform piece X in the style of Y.} Another series of researches done by Widmer at el., called PLCG \cite{46, 47, 48} uses data-mining to find rules for expressive performance. It's successor Phrase-decomposition/PLCG \cite{49} added hierarchical phrase structures support to the original PLCG system. And the latest research in the series called DISTALL \cite{50, 51} added hierarchical rules to the original one.

Most of the of the performance systems discussed above takes digitalized traditional musical notation (MusicXML etc.) or neutral audio as input. They have to figures out the expressive intention of the composer by musical analysis or assigned by the user. But another type of computer expressive performance has a great advantage over the previous ones, by combining computer composition and expressive performance, the performance module can share the performance intention directly with the composition module. Ossia \cite{61} and pMIMACS \cite{pmimacs} are two examples of this category.  This approach provides great possibility for creativity, but they can only play their own composition, which is rather limited.

\framebox{TODO:Fig.:selected figures from previous works}
\section{Additional Specialties}

Most expressive performance systems implicitly or explicitly generates piano performance, because it's relatively easy to collect training samples for piano and piano sound is relatively easy to synthesize. Yet, some systems generates music in other instruments, such as saxophone\cite{40, 41, 42}, trumpet\cite{24, 25}, flute \cite{39} and drums \cite{56}. These systems requires extra effort in creating instrument models in training, generation and synthesizing. 

If not specified, most systems handles traditional western tonal music. However, most saxophone-based works\cite{40, 41, 42} generates Jazz music, because saxophone is an iconic instrument in Jazz performance. And the Drumming System\cite{56} generates Brazilian drumming music.%The Bach Fugue System \cite{23}, literally, focus on fugue works composed by bach. 

Performing polyphonic music is much more challenging than monophonic music, because it requires synchronization between voices, while allowing each voice to have their own expression at the same time. Pop-E\cite{28} use a synchronization mechanism to achieve polyphonic performance. Bach Fugue System \cite{23} is created using the polyphonic rules in music theory about fugue, so it's inherently able to play polyphonic fugue. KCCA Piano System \cite{57}can generate homophonic music -- an upper melody with an accompaniment -- which is common in piano music.   Music Plus One \cite{52,53,54} is a little bit different because it's a accompaniment system, it adapts non-expressive orchestral accompaniment track to user's performance. Other systems usually generates monophonic tracks only. 

\framebox{REVIEW1}
