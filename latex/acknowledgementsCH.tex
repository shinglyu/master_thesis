\begin{acknowledgementsCH}

首先要感謝鄭士康教授,早在大三選修教授的專題時(感謝大學部導師張時中教授推薦),教授就給我完全的自由,讓我能夠慢慢培養出找題目、設計實驗、上台報告以及撰寫論文的能力。每週六的meeting教授也都會給予我非常實際且切中要點的建議。

感謝各位口試委員寶貴的建議,讓這份論文能夠更加完善。謝謝王真儀教授逐字逐句的幫我訂正論文並且提供非常專業的意見。感謝王育雯教授的樂理課為我的實驗打下的基礎。也感謝陳宏銘教授的多媒體訊號處理課讓我對電腦音樂有更深的認識。

感謝JCMG的各位學長姐與同學,特別是志鴻、御仁、鴻心、彥彬、韋安、鼎棋、晟文、傳佑、如江、廉喬,每次的討論都給我許多的靈感。

感謝振宇、俞仲、鍾愛、宗緯、廉喬、智展撥出你們寶貴的時間來幫我錄製實驗用的演奏範例,你們精湛的演出讓這個實驗可以有高品質的數據可以使用。

謝謝 Intel 在這兩年多來給予我經濟上的支持還有給予我學習的機會。感謝Allen Ouyang, Robinson Do, Chuanny Shiau三位主管讓我有機會參與各種富有挑戰性的計畫,也讓我可以很自由的調配上班與上學的時間。

感謝愛樂社的各位,特別是順德、芝潔、品臻、小女王、迪西、顧門口、俊麟、子恩、乃嘉、維中、子瑩、彥彤,愛樂社燃起了我對音樂的愛,與各位共度的音樂會時光也是經常是我研究的靈感來源。

另外要特別感謝台大音樂所的教授與同學,王育雯教授的樂理課的同學們在實驗中給了我許多的幫助與建議。也特別謝謝金立群教授給予我的諸多協助:WOCMAT上的批評指教、介紹口試委員、以及幫我轉貼網路問卷。

And I would like to thank all the contributors of the open source software community, especially the contributors of Python, music21, R, Rosegarden, Musescore, and Linux Mint Debian Edition. Without your great work, this thesis can't become a reality.

感謝我的家人在這20幾年來的支持與照顧,讓我可以無憂無慮的學習與成長,在我大學與研究所最忙碌的時候,家始終是我能夠最放鬆自在的地方。

最後要感謝我的女朋友,永遠笑臉迎人,為我帶來許多的歡笑,在我為研究忙的不可開交的時候也從來不會抱怨,總是默默的支持與鼓勵我。

這份研究若沒有諸位的協助是不可能完成的,在此我獻上我要獻上最誠摯的感謝,願耶和華賜恩予你與你全家。
   \flushright 呂行 謹誌 2014.6.10







\end{acknowledgementsCH}
