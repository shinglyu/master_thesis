\begin{acknowledgementsCH}
%走筆至此,這份研究也已經進行兩年多了,這份研究若沒有諸位的幫助是不可能完成的,在此我獻上我最誠摯的感謝。
%
%首先要感謝鄭士康教授,早在大三選修教授的專題時(感謝大學部導師張時中教授推薦),教授就給我完全的自由,讓我能夠慢慢培養出自己找題目、設計實驗、上台報告以及撰寫論文的能力。每週六的meeting教授也都會給予我非常實際且充滿啟發性的建議。
%
%也謝謝JCMG的各位學長姐與同學,特別是志鴻、御仁、鴻心、彥彬、韋安、鼎棋、晟文、傳佑、如江、廉喬,在每次的討論與交流,都給我許多的靈感。
%
%感謝振宇、俞仲、鍾愛、宗緯、廉喬、智展撥出他們寶貴的時間來幫我錄製實驗用的演奏範例,沒有你們精湛的演出,這個實驗只能淪為紙上談兵。
%
%謝謝 Intel 在這兩年多來給予我經濟上的支持還有給予我學習的機會。感謝Allen Ouyang, Robinson Do, Chuanny Shiau讓我有機會參與各式各樣有挑戰性的計畫,也讓我可以很自由的調配上班與上學的時間。
%
%感謝愛樂社的各位,不管是社課、音樂會都能和你們一起分享對音樂的愛,謝謝各位的友誼與照顧。
%
%另外要特別感謝台大音樂所的教授與同學,王育雯教授的樂理課為我的實驗打下了相當的基礎,各位一起上課的同學們在實驗中給了我許多的幫助與建議。也特別謝謝金立群教授給予我的諸多協助:WOCMAT上的批評指教、介紹口試委員、以及幫我轉貼網路問卷。
%
%And I would like to thank all contributors from the open source software community, especially the contributors of Python, music21, R, Rosegarden, Musescore, and Linux Mint Debian Edition. Without your great work, this thesis can't become a reality.
%
%感謝我的家人在這20幾年來的照顧與鼓勵,讓我可以無憂無慮的學習與成長,在我大學與研究所最忙碌的時候,家始終是我能夠最放鬆自在的地方。
%
%最後要感謝我的女朋友,永遠是笑臉迎人,為我帶來許多的歡笑,在我為研究忙的不可開交的時候也從來不會抱怨,總是默默的支持與鼓勵我。





\end{acknowledgementsCH}
