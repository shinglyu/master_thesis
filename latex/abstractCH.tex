\begin{abstractCH}

  電腦合成的音樂一向被認為是僵硬、機械化而且沒有音樂表現能力。因此能夠產生具有表現能力的電腦自動演奏系統將會對音樂產業、個人化娛樂以及表驗藝術領域有重大的影響。在這篇論文中,我們藉由隱藏式馬可夫模型結構的結構性支撐向量機(SVM-HMM)來設計一個可以產生具有表現能力音樂的電腦自動演奏系統。我們邀請六位研究生錄製了克萊門蒂(Muzio Clementi)的小奏鳴曲集 Op.36。我們手動將這些錄音分割成樂句,並且利用程式從中抽取出音樂特徵。這些音樂特徵藉由SVM-HMM訓練成數學模型後,可以利用這個數學模型來演奏訓練過程中沒有見過的樂譜(需要手動標注樂句)。此系統目前只能支援單音旋律。問卷調查的結果顯示,對於業餘或專業的音樂家來說,本系統產生的音樂尚不能達到真人的演奏水準,但是沒有音樂背景的受試者給予本系統產生的音樂的分數已經與真人演奏不相上下。
  
  關鍵字:電腦自動演奏、結構性支撐向量機、支撐向量機

\end{abstractCH}
