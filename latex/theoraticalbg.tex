%\chapter{Theoretical Background}
%\framebox{REVIEW1}
\section{A Brief Introduction to SVM-HMM}
\label{sec:svm-hmm}
In this thesis, we use the structural support vector machine to learn performance knowledge from expressive performance samples. Unlike the traditional SVM algorithm, which only produce univariate prediction, the structural SVM can produce structural predictions like trees, graphs or sequences. The structural SVM with hidden Markov model output (SVM-HMM) has been successfully applied to part-of-speech tagging problem\cite{svm2009}. There are some similarities between the part-of-speech tagging problem and the expressive performance problem. In the part-of-speech tagging, one tries to identify the role in which the word plays in the sentence, while in the expressive performance, one tries to determine how a note should be played, usually based on its role in the musical phrase. %For example, an authentic cadence at the end of a phrase is usually played louder and stronger than a embellishment note in the middle of a phrase. 
Thus, we believe the SVM-HMM will be a good candidate for expressive performance. The following introduction and formulas rely heavily on \cite{svm2009, svm2005, svm2003}.

%Ref: 20130420 slides

%TODO:discuss traditional SVM here?
The traditional SVM prediction problem can be described as finding a function 
$$h: \mathcal{X \rightarrow Y}$$ with lowest prediction error. $\mathcal{X}$ is the input features space, and $\mathcal{Y}$ is the prediction space. In a traditional SVM, elements in $\mathcal{Y}$ are labels (classification) or real values (regression). However, a structural SVM extends the framework to generate structural output, such as trees, graphs or sequences.
To extend SVM to support structured outputs, the problem is modified as finding a discriminant function $$F: \mathcal{X} \times \mathcal{Y} \rightarrow \mathcal{R}$$, in which the input/output pairs are mapped to a real number score. To predict an output $y$ for an input $x$, one try to maximize $F$ over all $y \in \mathcal{Y}$. 

$$f(x) = \argmax_{y\in\mathcal{Y}} F(w,x,y)$$

Let $F$ be a linear function of the following form:

$$ F = \mathbf{w}^{T}\Psi(x,y)$$, 
where $\mathbf{w}$ is the parameter vector, and $\Psi(x,y)$ is the kernel function relating input $x$ to output $y$. $\Psi$ can be defined to accommodate various kinds of structure. 

%emprical risk
For each structure we want to predict, a loss function that measures the accuracy of of a prediction is required. A loss function $\Delta:\mathcal{Y}\times\mathcal{Y}\rightarrow R$ needs to satisfy the following property:

$$\Delta(y, y') \geq 0 \ for\ y \neq y'$$
$$\Delta(y, y) = 0 $$

The loss function is assumed to be bounded. Let's assume the input-output pair $(x,y)$ is drawn from a join distribution P(x,y), the prediction problem is to minimize the total loss:

%TODO: total loss formula 2005 sec 2.1
$$R_p^\Delta = \int_{\mathcal{X} \times \mathcal{Y}} \Delta (y, f(x))dP(x,y)$$

Since we cannot directly find the distribution $P$, we need to replace this total loss with an empirical loss, which can be calculated from the observed training set of $(x_i, y_i)$ pairs.
%TODO: emprical loss
$$R_s^\Delta(f) = \frac{1}{n}\sum^n_{i=1}\Delta(y_i, f(x_i))$$

Now we are ready to extend SVM to structural output, starting with a linear separable case, and we will then extend it to a soft-margin formulation. 

A linear separable case can be expressed by a set of linear constrains
%TODO: 2005 formula 4
$$\forall i \in \{1,\cdots,n\}, \forall \hat{y_i}\in\mathcal{Y}: \mathbf{w}^T [\Psi(x_i, y_i) - \Psi(x_i, \hat{y_i})]\geq 0$$

The constrains imply that the groundtruth $y_i$ for $x_i$ has the minimum $F$ value than any other $\hat{y}_i \neq {y_i}$.

The key concept of SVM is the large margin principle. We not only want to find a solution that statisfies the constrains, but also we want to maximize the margin between the groundtruth and the second best $\hat{y}_i$:
%TODO: 2005 formula 4+
$$
\begin{aligned}
& \max_{\gamma, \mathbf{w}:\|\mathbf{w}\| = 1} \gamma \\
& s.t \; \forall i \in \{1,\cdots,n\}, \forall \hat{y_i} \in\mathcal{Y}: \mathbf{w}^T [\Psi(x_i, y_i) - \Psi(x_i, \hat{y_i})] \geq \gamma\\
\end{aligned}
$$

, which is equivalent to the convex quadratic programming problem:
%TODO: 2005 formula 5,k 6
$$
\begin{aligned}
   & \min_{\mathbf{w}, \xi_i \geq 0} \frac{1}{2}\|\mathbf{w}\|^2 \\\
    &s.t.\; \forall i \in \{1,\cdots,n\},\hat{y_i} \in \mathcal{Y}: \mathbf{w}^T[\Psi(x_i,y_i) - \Psi(x_i,\hat{y_i})] \geq 1\\
\end{aligned}
$$

To extend the linear-separable case to a non-separable case, slack variables $\xi_i$ can be introduced to penalize prediction errors, results in a soft-margin formalization:
%TODO: 2005 formula SVM1
$$
\begin{aligned}
   & \min_{\mathbf{w}, \xi_i \geq 0} \frac{1}{2}\|\mathbf{w}\|^2 + \frac{C}{n}\sum^n_{i=1}\xi_i\\
    &s.t.\; \forall i \in \{1,\cdots,n\},\hat{y_i} \in \mathcal{Y}: \mathbf{w}^T[\Psi(x_i,y_i) - \Psi(x_i,\hat{y_i})] \geq 1 - \xi_i \\
\end{aligned}
$$

$C$ is the weighting parameter controlling the trade-off between low training error and large margin. The optimal $C$ varies between different problems, so experiments should be conducted to find the optimal $C$ for our problem.

Intuitively, a constrain violation with a larger loss should be penalized more than the one with a smaller loss. So I. Tsochantaridis et al. \cite{svm2005} proposed two possible way to take the loss function into account. The first way is to re-scale the slack variable by the inverse of the loss, so a high loss leads to a smaller re-scaled slack variable:
%slack rescaling

$$
\begin{aligned}
   & \min_{\mathbf{w}, \xi_i \geq 0} \frac{1}{2}\|\mathbf{w}\|^2 + \frac{C}{n} \sum^n_{i=1}\xi_i\\
    &s.t.\; \forall i \in \{1,\cdots,n\},\hat{y_i} \in \mathcal{Y}: \mathbf{w}^T[\Psi(x_i,y_i) - \Psi(x_i,\hat{y_i})] \geq 1 - \frac{\xi_i}{\Delta(y_i, \hat{y_i})} \\
\end{aligned}
$$

The second way is to re-scale the margin, which yields 
%margin-rescaling
$$
\begin{aligned}
   & \min_{\mathbf{w}, \xi_i \geq 0} \frac{1}{2}\|\mathbf{w}\|^2  + \frac{C}{n} \sum^n_{i=1}\xi_i\\
    &s.t.\; \forall i \in \{1,\cdots,n\},\hat{y_i} \in \mathcal{Y}: \mathbf{w}^T[\Psi(x_i,y_i) - \Psi(x_i,\hat{y_i})] \geq \Delta(y_i, \hat{y_i}) - \xi_i\\
%
\end{aligned}
$$
But the above quadratic programming problem has a very large number ($O(n|\mathcal{Y}|)$) of constrains, which will take considerable time to solve. I. Tsochantaridis et al. \cite{svm2005} proposed a greedy algorithm to speed up the process by selecting only part of the constrains that contributes the most to finding the solution. Initially, the solver starts with an empty working set containing no constrains. Than the solver iteratively scans the training set to find the most violated constrains under the current solution. If a constrain is violated more times than a desired threshold, the constrain is added to the working set of constrains. Then the solver re-calculates the solution under the new working set. The algorithm will terminate once no more constrain can be added under the desired precision.

In a later work by Joachims et al.\cite{svm2009}, they created a new formulation and algorithm to further speed up the algorithm. Instead of using one slack variable for each training sample, which resulting in a total of $n$ slack variables, they use a single slack variable for all $n$ training samples. The following formula is the 1-slack version of slack-rescaling structural SVM:
%1-slack
$$
\begin{aligned}
    & \min_{\mathbf{w}, \xi_i \geq 0} \frac{1}{2}\|\mathbf{w}\|^2  + C \xi\\
    &s.t.\; \forall i \in \{1,\cdots,n\},\hat{y_i} \in \mathcal{Y}: \mathbf{w}^T[\Psi(x_i,y_i) - \Psi(x_i,\hat{y_i})] \geq \frac{1}{n}\sum^n_{i=1}1 - \frac{\xi}{\Delta(y_i, \hat{y_i})} \\
\end{aligned}
$$

And margin-rescaling structural SVM:

$$
\begin{aligned}
    & \min_{\mathbf{w}, \xi_i \geq 0} \frac{1}{2}\|\mathbf{w}\|^2  + C \xi\\
    & s.t.\; \forall i \in \{1,\cdots,n\},\hat{y_i} \in \mathcal{Y}: \mathbf{w}^T[\Psi(x_i,y_i) - \Psi(x_i,\hat{y_i})] \geq \frac{1}{n}\sum^n_{i=1}\Delta(y_i, \hat{y_i}) - \xi \\
\end{aligned}
$$
%                $$\min_{\mathbf{w}, \xi_i \geq 0} \frac{1}{2}\mathbf{w}^T\mathbf{w} + \frac{C}{n} \sum_{i=1}^{n} \xi_i$$
%                s.t. for $i = 1\cdots n$
%                $$\forall \hat{y_i} \in \mathcal{Y}: \mathbf{w}^T[\Psi(x_i,y_i) - \Psi(x_i,\hat{y_i})] \geq \Delta(y_i, \hat{y_i}) - \xi_i $$
%
%                $$\forall \hat{y_i} \in \mathcal{Y}: \mathbf{w}^T[\Psi(x_i,y_i) - \Psi(x_i,\hat{y_i})] \geq 1 - \frac{\xi+i}{\Delta(y_i, \hat{y_i})}$$
Detailed proofs on how the new formulation is equally general as the old one is given in the paper \cite{svm2009}.

With the framework described above, the only problem left is how to define the general loss function and $\Psi$.  Drawing the inter-state dependencies and time dependencies concept from hidden Markov model, Y. Altun et al.\cite{svm2003} proposed two types of features for an equal-length observation/label sequence pair $(x,y)$. The first is the interaction of a observed feature $x^s$ with a label $y^t$, the other is the interaction between neighboring labels $y^s$ and $y^t$. 

%   \begin{figure*}[tp]
%      \begin{center}
%         \includegrapsics[width=0.8\textwidth]{fig/TBDFigure}
%      \end{center}
%      \caption{Hidden Markov Model}
%      \label{fig:hmm}
%   \end{figure*}

To illustrate the method, we use an example from music: for some observed features $\Psi_r(x^s)$ of a note $x$ located in $s$-th position of the phrase, and assume $\left[ \left[ y^t = \tau \right] \right]$ denotes the $t$-th note is played at a velocity of $\tau$, the interaction of the observed feature and the label can be written as:
%TODO hmm 3 formula 4
$$\psi^{st}_{r\sigma}(\mathbf{x}, \mathbf{y}) = \left[\left[y^t = \tau \right] \right]\Psi_r(x^s),\; 1\leq\gamma\leq d,\; \tau \in \Sigma $$

And the interaction between labels can be written as:
%TODO hmm 3 formula 5
$$\hat{\psi}^{st}_{r\sigma}(\mathbf{x}, \mathbf{y}) = \left[\left[y^s = \sigma \wedge y^t = \tau \right] \right],\; \sigma, \tau \in \Sigma $$

By selecting an order of dependency for the HMM model, we can further restrict $s$'s and $t$'s. For example, for a first-order HMM, $s = t$ for the first feature, and $s = t-1$ for the second feature. The two features on the same time $t$ is then stacked into a vector $\Psi(x,y;t)$. The feature map for the whole sequence is simply the sum of all the feature vectors 

%TODO hmm 3 formula 6
$$\Psi(\mathbf{x}, \mathbf{y}) = \sum^T_{t=1}\Psi(\mathbf{x}, \mathbf{y};t)$$

The distance, i.e. the general loss function, between two feature maps depends on the number of common label segments and the inner product between the input features sequence with common labels.


$$\Delta(\Psi(\mathbf{x}, \mathbf{y}), \Psi(\mathbf{\hat{x}}, \mathbf{\hat{y}})) = \sum_{s,t}\left[\left[y^{s-1} = \hat{y}^{t-1}\wedge y^s = \hat{y}^t\right] \right] + \sum_{s,t}\left[\left[y^{s} = \hat{y}^{t}\right] \right]k(x^s, \hat{x}^t)$$

Finally, during the prediction process, a Viterbi-like decoding algorithm is used to effeciently find a $y$ that maximize $F$.


%TODO: how to define loss function for HMM?
%(2003) section 3
%ovserved output <--> tag
%previous tag <--> this tag (1-order markov)
% Psi = each note's above two property summed up
%similarity = same prev tag <--> this tag sequence + same tag <--> observed output distance

%Hard-margin one
%Soft-margin one => introduce slack variable 
%Example with large loss should be emphasized => slack rescaling
%Margin can also be scaled => margin rescaling
%There are too many constrains => greedy algo (2005), select a subset of constrains from the most violated constrains to solve
%To speed up, n-slack variables are reduced to 1-slack variable (2009)

%           \item Prediction error (risk):
%               $$R^\Delta_p(h) = \int_{\mathcal{X}\times\mathcal{Y}}\Delta(y, h(x)) dP(x,y)$$
%               \begin{tabular}{ll}
%                   where & $\Delta()$ is the loss function \footnote{Must satisfy $\Delta(x,x) = 0$, $\Delta(x,y) > 0$}\\
%                   & P(x,y) is the joint distribution of $\mathcal{X}$ and $\mathcal{Y}$
%               \end{tabular}

%    \begin{frame}{Emperical Risk}
%       \begin{itemize}
%           \item Emperical Risk from training sample $S$:\footnote{Emperical Risk Minimization Priciple (Vapnik V (1998) Statistical Learning Theory. Wiley, Chichester, GB)}

%               $$R^\Delta_S(h) = \frac{1}{n}\sum_{i=1}^{n}\Delta(y_i, h(x_i))$$
%                   where  $\Delta()$ is the loss function 

%           \item Classification SVM

%                   $$\displaystyle \min_{\mathbf{w}, \xi_i \geq 0} \frac{1}{2}\mathbf{w}^T\mathbf{w} + \frac{C}{n} \sum_{i=1}^{n} \xi_i$$
%                   s.t. $$\forall i\in {1,\cdots n}: y_i (\mathbf{w}^T x_i) \geq 1-\xi_i$$
                  

%           \item Learn a discriminant function $f:\mathcal{X} \times \mathcal{Y} \rightarrow \Re$ 
%           \item Given $x$, maximizing $f$ over all $y \in \mathcal{Y}$
%               $$h_\mathbf{w} (x) = \argmax_{y\in\mathcal{Y}} f_\mathbf{w} (x,y)$$
%           \item 
%               in which $$f_\mathbf{w} (x,y) = \mathbf{w}^T{\Psi}(x,y)$$
%               \begin{tabular}{ll}
%                   where & $\mathbf{w} \in \Re^N$ is a parameter vector\\
%                         & $\Psi(x,y)$ is a feature vector relating $x$ and $y$
%               \end{tabular}
              

 
% \section{Structural SVM}

% \section{Theoretical Details}
% %&=& &=& &=& &=& &=& &=& &=& &=& =
%    \begin{frame}{Lowest Risk}
%       \begin{itemize}
%           \item Prediction error (risk):
%               $$R^\Delta_p(h) = \int_{\mathcal{X}\times\mathcal{Y}}\Delta(y, h(x)) dP(x,y)$$
%               \begin{tabular}{ll}
%                   where & $\Delta()$ is the loss function \footnote{Must satisfy $\Delta(x,x) = 0$, $\Delta(x,y) > 0$}\\
%                   & P(x,y) is the joint distribution of $\mathcal{X}$ and $\mathcal{Y}$
%               \end{tabular}

%           %\item Training sample: $(x_1, y_1), (x_2, y_2), \cdots$ where $y_i$'s may have structural relationship
              
%       \end{itemize}
%    \end{frame}
%    \begin{frame}{Emperical Risk}
%       \begin{itemize}
%           \item Emperical Risk from training sample $S$:\footnote{Emperical Risk Minimization Priciple (Vapnik V (1998) Statistical Learning Theory. Wiley, Chichester, GB)}

%               $$R^\Delta_S(h) = \frac{1}{n}\sum_{i=1}^{n}\Delta(y_i, h(x_i))$$
%                   where  $\Delta()$ is the loss function 

%           %\item Training sample: $(x_1, y_1), (x_2, y_2), \cdots$ where $y_i$'s may have structural relationship
              
%       \end{itemize}
%    \end{frame}

%    \begin{frame}{Traditional SVM}
%       \begin{itemize}
%           \item Classification SVM

%                   $$\displaystyle \min_{\mathbf{w}, \xi_i \geq 0} \frac{1}{2}\mathbf{w}^T\mathbf{w} + \frac{C}{n} \sum_{i=1}^{n} \xi_i$$
%                   s.t. $$\forall i\in {1,\cdots n}: y_i (\mathbf{w}^T x_i) \geq 1-\xi_i$$
                  

              
%       \end{itemize}
%    \end{frame}

%    \begin{frame}{Structural SVM}
%       \begin{itemize}
%           \item Extend SVM for structural output
%           \item Learn a discriminant function $f:\mathcal{X} \times \mathcal{Y} \rightarrow \Re$ 
%           \item Given $x$, maximizing $f$ over all $y \in \mathcal{Y}$
%               $$h_\mathbf{w} (x) = \argmax_{y\in\mathcal{Y}} f_\mathbf{w} (x,y)$$
%           \item 
%               in which $$f_\mathbf{w} (x,y) = \mathbf{w}^T{\Psi}(x,y)$$
%               \begin{tabular}{ll}
%                   where & $\mathbf{w} \in \Re^N$ is a parameter vector\\
%                         & $\Psi(x,y)$ is a feature vector relating $x$ and $y$
%               \end{tabular}


                  

              
%       \end{itemize}
%    \end{frame}

%    \begin{frame}{N-slack Formulations}
%       \begin{itemize}
%           \item margin-rescaling: change hinge, fixing slope
%              $$\Delta_{MR}(y,h_\mathbf{w}) = \max_{\hat{y} \in \mathcal{Y}} \{ \Delta(y, \hat{y}) - \mathbf(x)^T {\Psi}(x,y) + \mathbf{w}^T{\Psi}(x,\hat{y}\} \geq \Delta(y,h_\mathbf{w}(x))$$
%           \item slack-rescaling: fixing hinge, changing slope
%              $$\Delta_{SR}(y,h_\mathbf{w}) = \max_{\hat{y} \in \mathcal{Y}} \{ \Delta(y, \hat{y}) (1 - \mathbf(x)^T {\Psi}(x,y) + \mathbf{w}^T{\Psi}(x,\hat{y} )\} \geq \Delta(y,h_\mathbf{w}(x))$$
              
%       \end{itemize}
%    \end{frame}

%    \begin{frame}{Optimization Problems}
%       \begin{itemize}
%           \item
%                $$\displaystyle \min_{\mathbf{w}, \xi_i \geq 0} \frac{1}{2}\mathbf{w}^T\mathbf{w} + \frac{C}{n} \sum_{i=1}^{n} \xi_i$$
%                s.t. for $i = 1\cdots n$
%           \item n-slack structural SVM w/ margin-rescaling
%                $$\forall \hat{y_i} \in \mathcal{Y}: \mathbf{w}^T[\Psi(x_i,y_i) - \Psi(x_i,\hat{y_i})] \geq \Delta(y_i, \hat{y_i}) - \xi_i $$

%           \item n-slack structural SVM w/ slack-rescaling
%                $$\forall \hat{y_i} \in \mathcal{Y}: \mathbf{w}^T[\Psi(x_i,y_i) - \Psi(x_i,\hat{y_i})] \geq 1 - \frac{\xi+i}{\Delta(y_i, \hat{y_i})}$$
%       \end{itemize}
%    \end{frame}

%    \begin{frame}{1-Slack Formulation}
%       \begin{itemize}
%           \item
%       \end{itemize}
%    \end{frame}


%TODO: theoratical background


%\section{Brent's Method}
%\section{diff}
