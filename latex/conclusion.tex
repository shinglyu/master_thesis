\chapter{Conclusions}
\framebox{REVIEW1}
\framebox{TODO:summary}
We have created a system that can create expressive music performance from monophonic score notation (with annotated phrasing), based on the SVM-HMM algorithm. We have also created a corpus consisting of score and MIDI recordings from which the system can learn performance knowledge. From our subjective test, we show that although the amateur and professional musician can still differentiate the generated performance from human recordings, subjects with no music background is already giving equal ratings to the generated performance and human recordings.

There are many room for improvements. Structural expressions such as phrasing, contract between sections, or even contrast between movements can be added. Other information like text notations, harmonic analysis and musicological analysis can be added. Supporting homophonic or polyphonic music is also important for the system to be useful. Sub-note expressions like physical model synthesizer or envelope shaping can also be applied to generate performances for specific musical instruments.  It's also crucial to test the system on more samples from different genre or music style. We also believe combining rule-based model and machine learning model may be a possible direction for computer expressive music performance research, because machine learning methods can learn subconscious expressions, while rule based system can serve as a high level guideline for structural expression.User can also control the overall expression easily by tweaking the rules.
\framebox{TODO:error model}


%TODO: structure
%TODO Expert system
%TODO Model selection
%TODO Jazz style
%TODO
