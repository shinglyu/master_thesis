\chapter{Conclusions}
We have created a system that can perform monophonic score expressively. The expressive performance knowledge is learned from hum an recording using structural support vector machine with hidden Markov model output (SVM-HMM). We have also created a corpus consisting of scores and MIDI recordings. From our subjective test, we show that although the amateur and professional musician can still differentiate the generated performance from human recordings, test subjects with no music background are giving equal ratings to the generated performance and human recordings, which means our system has the same expressive power as human.

There are many room for improvement. Structural expressions such as phrasing, contrast between sections, or even contrast between movements can be added, which requires automatic structural analysis. Other information like text notations, harmonic analysis and other musicological analysis can be added to the learning process. Supporting homophonic or polyphonic music is also important for the system to be useful. Sub-note expressions like physical model synthesizer or envelope shaping can also be applied to generate performances for specific musical instruments. It's also crucial to test the system on more samples of different genre or music style. We also believe that combining rule-based model and machine-learning model may be a possible direction for computer expressive music performance research. With rules serving as a high level guideline for structural expression, the machine-learning model can focus on note or sub-note level expression. User can gain more control by tweaking the rules.
%\framebox{TODO:error model}


%TODO: structure
%TODO Expert system
%TODO Model selection
%TODO Jazz style
%TODO
