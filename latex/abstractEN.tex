\begin{abstractEN}

Computer generated music is known to be robotic and inexpressive. A computer system that can generate expressive performance potentially has significant impact on music production industry, personalized entertainment or even art. In this paper, we have designed and implemented a system that can generate expressive performance using structural support vector machine with hidden Markov model output (SVM-HMM). We recorded six sets of Muzio Clementi's Sonatina Op.36 performed by six graduate students. The recordings and scores are manually split into phrases and had their musical features automatically extracted. Using the SVM-HMM algorithm, a mathematical model of expressive performance knowledge is learned from these features. The trained model can generate expressive performances for previously unseen scores (with user-assigned phrasings). The system currently supports monophonic music only. Subjective test shows that the computer generated performances still cannot achieve the same level of expressiveness of human performers, but quantitative similarity measures show that the computer generated performances are much similar to human performances than inexpressive MIDIs.

Keywords: Computer Expressive Performance, Performance Rendering, Structural SVMs, Support Vector Machines.
\end{abstractEN}
