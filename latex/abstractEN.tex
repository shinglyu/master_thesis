\begin{abstractEN}

Computer generated music is known to be robotic and inexpressive. A computer system that can generate expressive performance can potentially have significant impact on music production, personalized entertainment or even art. In this paper, we have designed and implemented a system that can generate expressive performance using structural support vector machine with hidden Markov model output (SVM-HMM). We recorded six sets of Muzio Clementi's Sonatina Op. 36 performed by six graduate students. The recordings and scores are manually split into phrases. Their musical features are automatically extracted. Using the SVM-HMM algorithm, a mathematical model of performance knowledge is learned from these features. The trained model can generate expressive performances for previously unseen scores (with manually-assigned phrasing). The system currently supports monophonic music only. Subjective test shows that for amateur and professional musician, the generated performance still need improvements to be comparable to human recording, but the generated performance received nearly the same rating as human recordings from participants without music background. 

Key words: Computer Expressive Performance, Performance Rendering, Structural SVMs, Support Vector Machines.
\end{abstractEN}
