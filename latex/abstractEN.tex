\begin{abstractEN}

Computer generated music is known to be robotic and unexpressive. %Many efforts have been devoted to generate computer systems that can perform music expressively. 
A computer system that can generate expressive performance will have significant impact on music produciton, personalized entertainment or even art. In this paper, we have designed and implemented a system that can generate expressive performance using the Structural Support Machine with Hidden Markove Model output algorithm (SVM-HMM). We recorded 6 sets of Muzio Clementi's Sonatina Op.36 performed by 6 graduate students. The recordings and scores are manually split into phrases. Their musical features are automatically extracted by the program. Using the SVM-HMM algorithm, a mathmatical model is trained from these features. The trained model can then be used to generate expressive performance for previously unseen scores (with manually-assigned phrasing). The system currently supports monophonic music only. Subjective test shows that for amateur and professional musician, the generated performance still need improvements to be comparable to huamn recording, but the generated performance is equally favorable as human recordings for participants without music background. 
Key words: Computer Expressive Performance, Performance Rendering, Structural SVMs, Support Vector Machines.

\end{abstractEN}
