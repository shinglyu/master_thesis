\chapter{Experiments and Results}
\framebox{TODO:general introduction to the experiments}
\section{Onset Timing Normalization Selection}
\subsection{Experiment Design}
In section \ref{TODO:feature section}, four nomalization method is proposed. To see which one is most robust, we validated them with emprical data. A robust normalization method should produce a mean-reverse sequence, not a monotonic increase or decrease sequence. Four types of nomalization method is applied to every samples in the corpus, the result is shown in Figure \ref{fig:normalize_result} A regression analysis was applied to the result to see if there are clear trend in the normalized value.
\subsection{Results}

\section{Parameter Selection}
\subsection{Experiment Design}
Structural Support Vector Machine has some parameters that needed to be adjusted. We will leave the others to the defaults and change the SVM C trad-off parameter in this experiment. Since three models are learned for three performance features, we have three parameters to adjust. 
%TODO default parameters

[TODO: phrases count] phrases from [TODO:song counts] songs are used for training. Every first, fifth, and tenth phrases from each song is not included in the training sample, but used as testing samples. A three-by-three grid is layed out for three C parameters, each C takes the value of the powers of tenfrom [TODO: Cs] $10^{-5}$ to $10^4$, so [TODO: num of experiment] paramenters are tested. Then the result is validated
\begin{enumerate}
	\item Are all the output samples successfuly generated? (Generation may fail if the performance features are unreasonable, for example, negative onset timeing.)
	\item Is the order of the notes preserved? Sometimes the first note is delayed too long and the second note is played too early, so the order is swaped.
	\item Are there any extreme parameters that makes the expressive performance unnatural?
\end{enumerate}

The first two criterias are checked by python scripts, the last one is done by manual inspection.


\subsection{Results}
\framebox{TODO:experiment result}

\section{Turing Test -- Human-like}
\framebox{TODO:experiment result}
\subsection{Experiment Design}
\framebox{TODO:experiment result}

\subsection{Results}
\framebox{TODO:experiment result}

\section{Turing Test -- Style}
\framebox{TODO:experiment result}
\subsection{Experiment Design}
\framebox{TODO:experiment result}

\subsection{Results}
\framebox{TODO:experiment result}
